\documentclass[10pt, aspectratio=169]{beamer}

% Theme
\usetheme{metropolis}
\usecolortheme{default}
\metroset{sectionpage=none} % Disable separate section title slides

% Packages
\usepackage{booktabs}
\usepackage{graphicx}
\usepackage{geometry}
\usepackage{tikz}

% Logo on every slide (top right)
\addtobeamertemplate{frametitle}{}{%
    \begin{tikzpicture}[remember picture,overlay]
        \node[anchor=south west, yshift=2pt] at (current page.south west) {\includegraphics[height=1.2cm]{images/bkc_logo.png}};
    \end{tikzpicture}
}

% Metadata
\title{Internship Report}
\subtitle{Network Operations \& ISP Support at Subisu Cablenet Pvt. Ltd.}
\author{Pratham Khanal}
\date{\today}
\institute{Tribhuvan University \\ Institute of Science and Technology}

\begin{document}

% Title Slide
\maketitle

% Outline
\begin{frame}{Outline}
    \tableofcontents[hideallsubsections]
\end{frame}

% 1. Introduction
\section{1. Introduction}
\begin{frame}{Introduction}
    \begin{itemize}
        \item \textbf{Role}: Network Operations and ISP Support Intern.
        \item \textbf{Organization}: Subisu Cablenet Pvt. Ltd. (Chitwan Branch).
        \item \textbf{Duration}: 10 Weeks (Kartik 2082 -- Poush 2082).
        \item \textbf{Overview}:
        \begin{itemize}
            \item Bridged the gap between theoretical knowledge and professional practice.
            \item Work conducted in a 24/7 Network Operations Center (NOC).
            \item Focused on monitoring, physical layer connectivity, and system administration.
        \end{itemize}
    \end{itemize}
\end{frame}

% 2. Problem Statement
\section{2. Problem Statement}
\begin{frame}{Problem Statement}
    Isolating and resolving real-world ISP challenges:
    \vspace{0.5cm}
    \begin{itemize}
        \item \textbf{Signal Attenuation}: High loss due to dust or bent fiber cables (Low Rx power).
        \item \textbf{Network Congestion}: High latency and packet drop during peak hours (7 PM -- 10 PM).
        \item \textbf{Connectivity Disputes}: IP conflicts and "Loss of Signal" (LOS) troubleshooting.
        \item \textbf{Security Threats}: Unauthorized access attempts and DDoS risks on the core network.
    \end{itemize}
\end{frame}

% 3. Objectives
\section{3. Objectives}
\begin{frame}{Objectives}
    The primary goal was to gain diverse hands-on experience in a large-scale network.
    \vspace{0.5cm}
    \begin{itemize}
        \item \textbf{Technical Exposure}: Understand infrastructure, connectivity standards, and troubleshooting.
        \item \textbf{Operational Support}: Assist in system monitoring (Zabbix) and administrative workflows.
        \item \textbf{Skill Development}: Master configuration of Layer-2 switches and Linux-based servers.
    \end{itemize}
\end{frame}

% 4. Projects Involved
\section{4. Projects Involved}
\begin{frame}{Projects Involved}
    \subsection{FTTH Signal Optimization}
    \textbf{Project 1: FTTH Signal Optimization and Diagnostics}
    \begin{itemize}
        \item \textbf{Goal}: Reduce "Loss of Signal" (LOS) tickets.
        \item \textbf{Action}: Analyzed OLT power metrics; identified customers with $<-27$ dBm signals.
        \item \textbf{Outcome}: Guided technicians to clean connectors and replace bent patch cords.
    \end{itemize}

    \vspace{0.5cm}

    \textbf{Project 2: Network Security Implementation}
    \begin{itemize}
        \item \textbf{Goal}: Harden network against unauthorized access.
        \item \textbf{Action}: Deployed Standard and Extended ACLs on routers.
        \item \textbf{Outcome}: Blocked malicious IP ranges and enforced "Deny All" default policies.
    \end{itemize}
\end{frame}

% 5. Activities Performed
\section{5. Activities Performed}
\begin{frame}{Activities Performed}
    \begin{columns}[T]
        \begin{column}{0.48\textwidth}
            \textbf{Network Monitoring}
            \begin{itemize}
                \item Used \textbf{Zabbix} to monitor uptime, latency, and throughput.
                \item Identified critical outages and power failures.
            \end{itemize}

            \vspace{0.3cm}

            \textbf{Device Configuration}
            \begin{itemize}
                \item Configured Cisco Layer-2 switches.
                \item Created VLANs: \textbf{10} (Data) and \textbf{20} (Voice).
            \end{itemize}
        \end{column}

        \begin{column}{0.48\textwidth}
            \textbf{Linux System Admin}
            \begin{itemize}
                \item SSH into CentOS servers.
                \item Log analysis with \texttt{grep}, \texttt{tail}.
                \item DNS service status checks.
            \end{itemize}

            \vspace{0.3cm}

            \textbf{Fiber Diagnostics}
            \begin{itemize}
                \item Splicing with fusion splicers.
                \item Differentiating Single-mode vs. Multi-mode.
            \end{itemize}
        \end{column}
    \end{columns}
\end{frame}

% 6. Conclusion
\section{6. Conclusion}
\begin{frame}{Conclusion \& Learning Outcomes}
    \textbf{Conclusion}
    \begin{itemize}
        \item Successfully transitioned from classroom theory to live industry practice.
        \item Gained confidence in handling enterprise-grade equipment (Cisco, OLTs).
    \end{itemize}

    \vspace{0.3cm}

    \textbf{Key Learning Outcomes}
    \begin{itemize}
        \item \textbf{Technical}: Fiber optics mastery, VLAN/Subnetting, Linux CLI usage.
        \item \textbf{Professional}: Incident lifecycle management (CRM), Documentation, and Pressure handling.
    \end{itemize}
\end{frame}

\begin{frame}[standout]
    Thank You!
    \vspace{1cm}

    \small Questions?
\end{frame}

\end{document}
