\documentclass[12pt, a4paper]{article}
\usepackage[top=1in, bottom=1in, left=1in, right=1in]{geometry}
\usepackage{mathptmx}
\usepackage{setspace}
\usepackage{titlesec}
\usepackage{enumitem}
\usepackage{fancyhdr}
\usepackage{longtable}
\usepackage{array}

\pagestyle{empty}

\newcommand{\weeklylogentry}[6]{
    \newpage
    \begin{center}
        \Large \textbf{INTERNSHIP WEEKLY PROGRESS REPORT}
    \end{center}
    \vspace{0.2cm}

    \noindent \rule{\textwidth}{1pt}
    \vspace{0.3cm}

    \noindent \textbf{Student Name:} John Doe \hfill \textbf{Week Number:} #1 \hfill \textbf{Date:} #2

    \vspace{0.2cm}
    \noindent \rule{\textwidth}{1pt}
    \vspace{0.4cm}

    \noindent \large \textbf{1. Responsibility / Main Focus:} \\
    \normalsize #3
    \vspace{0.4cm}

    \noindent \large \textbf{2. Activities Performed:}
    \normalsize
    \begin{itemize}[leftmargin=*, parsep=2pt, itemsep=2pt]
    #4
    \end{itemize}
    \vspace{0.2cm}

    \noindent \large \textbf{3. Observations / Learning:} \\
    \normalsize #5
    \vspace{0.4cm}

    \noindent \large \textbf{4. Plan for Next Week:} \\
    \normalsize #6
    \vspace{0.4cm}

    \noindent \rule{\textwidth}{1pt}
    \vspace{0.3cm}

    \noindent \large \textbf{5. Performance Appraisal by Mentor:}
    \normalsize
    \vspace{0.2cm}

    \vspace{2.5cm} % Space for handwritten comments

    \noindent \rule{\textwidth}{0.5pt}
    \vspace{0.5cm}

    \noindent \begin{tabular}{@{}p{0.45\textwidth}p{0.1\textwidth}p{0.45\textwidth}@{}}
        \textbf{Supervisor Name:} & & \textbf{Mentor Name:} \\[1.5cm]
        \rule{0.8\linewidth}{0.5pt} & & \rule{0.8\linewidth}{0.5pt} \\
        Signature & & Signature \\[0.2cm]
        Date: \rule{0.5\linewidth}{0.5pt} & & Date: \rule{0.5\linewidth}{0.5pt} \\
    \end{tabular}
}

\begin{document}

% Week 1
\weeklylogentry{First}{2082/07/23}
{Onboarding and Environment Setup}
{
    \item Attended internship orientation and introduction to the Development team.
    \item Set up the development environment, including PHP, MySQL, Node.js, and VS Code.
    \item Studied the existing project requirements for "HisabKitab" and established a project roadmap.
}
{Setting up a proper development environment is crucial for workflow efficiency and consistency across the team.}
{Start working on frontend wireframes and HTML structure.}

% Week 2
\weeklylogentry{Second}{2082/07/30}
{Frontend Design and Wireframing}
{
    \item Focused on fundamental web technologies: HTML5 and CSS3.
    \item Designed the initial wireframes and UI components for the HisabKitab dashboard.
    \item Implemented responsive layouts using CSS Flexbox and Grid.
}
{Learned that CSS Flexbox and Grid offer powerful tools for creating responsive designs that adapt to different screen sizes.}
{Add interactivity to the UI using JavaScript.}

% Week 3
\weeklylogentry{Third}{2082/08/07}
{Frontend Logic Implementation}
{
    \item Introduced JavaScript for frontend interactivity.
    \item Implemented dynamic DOM manipulation for adding and removing expense rows in the UI.
    \item Learned about asynchronous JavaScript (Promises and Fetch API) for future data fetching.
}
{Direct DOM manipulation allows for dynamic user experiences without page reloads, which significantly improves the perceived speed of the application.}
{Begin server-side programming with basic PHP.}

% Week 4
\weeklylogentry{Fourth}{2082/08/14}
{Backend Basics and Security}
{
    \item Transitioned to server-side development with basic PHP.
    \item Built simple script-based CRUD operations for managing a mock customer list.
    \item Understood the importance of server-side validation and security.
}
{Server-side validation is essential for security as client-side checks can be bypassed. Input sanitization is key to preventing injection attacks.}
{Install and explore the Laravel framework.}

% Week 5
\weeklylogentry{Fifth}{2082/08/21}
{Framework Setup and Routing}
{
    \item Installed and configured the Laravel framework.
    \item Explored the Laravel directory structure and primary configuration files.
    \item Set up the first set of routes and successfully rendered a "Hello World" view.
}
{Using a framework like Laravel provides a structured and secure foundation for application development.}
{Implement Controllers and master Views using Blade.}

% Week 6
\weeklylogentry{Sixth}{2082/08/28}
{MVC Architecture Implementation}
{
    \item Deep dive into the MVC (Model-View-Controller) architecture.
    \item Created the first Controller and started migrating frontend assets into Blade templates.
    \item Implemented a master layout file for consistent UI across all pages.
}
{Blade templates significantly reduce code duplication by allowing the use of master layouts and reusable components.}
{Integrate real database connections and design migrations.}

% Week 7
\weeklylogentry{Seventh}{2082/09/06}
{Database Design and Seeding}
{
    \item Focused on database integration and migrations.
    \item Created the \texttt{Customer} and \texttt{Transaction} models and their corresponding migrations.
    \item Seeded the database with sample data using Laravel Factories.
}
{Database migrations serve as version control for the database schema, making it easy to share and synchronize database structures among team members.}
{Develop core CRUD functionality for transactions.}

% Week 8
\weeklylogentry{Eighth}{2082/09/13}
{Core Functionality Development}
{
    \item Developed core CRUD modules for the Transaction management system.
    \item Implemented Eloquent relationships (One-to-Many) between Customers and Transactions.
    \item Added form validation and custom error messages in Blade views.
}
{Eloquent ORM relationships simplify complex SQL joins into readable method calls, speeding up the development of related data queries.}
{Implement search functionality via internal APIs.}

% Week 9
\weeklylogentry{Ninth}{2082/09/20}
{API Development and Optimization}
{
    \item Developed a search API for customers to improve user efficiency.
    \item Integrated the Search API with the frontend using JavaScript and Fetch.
    \item Performed performance optimization of database queries using eager loading (\texttt{with()}).
}
{Eager loading is critical for performance optimization to prevent the N+1 query problem when fetching related data models.}
{Final testing, bug fixing, and report compilation.}

% Week 10
\weeklylogentry{Tenth}{2082/09/27}
{Testing, Documentation, and Review}
{
    \item Final testing and bug fixing of the complete "HisabKitab" application.
    \item Compiled the weekly logs into the final internship report.
}
{Thorough testing helps identify edge cases that might be missed during initial development. Documentation is vital for the long-term maintainability of the project.}
{Final submission and project handover.}

\end{document}
