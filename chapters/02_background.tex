\chapter{Organization Details and Literature Review}
\section{Introduction to Organization}

In the 21st century, Information Technology (IT) plays a vital role in the survival, growth, and economic development of industries and nations. An Internet Service Provider (ISP) is central to this ecosystem, providing access to the internet through technologies such as copper wires, wireless connections, and optical fiber. ISPs employ various distribution methods including Dial-up, DSL, ADSL, broadband wireless, cable modems, and fiber optics to connect clients, corporate houses, and general customers. In the context of Nepal, there are more than forty ISPs, each utilizing distinct distribution methods.

The internship was conducted at Subisu Cablenet, a leading organization in this sector. As IT continues to advance, organizations must close the gap between business needs and delivery capabilities to move beyond operational fixes and plan for the future.

\begin{itemize}
    \item \textbf{Vision}: Subisu’s vision is to transform ideas and innovation into viable and creative solutions by leveraging the strength of state-of-the-art technology and by nurturing and harnessing the skills of well-trained, motivated, and committed teams contributing to the Social and Economic development phenomena of the Kingdom.
    \item \textbf{Mission}: To be a leading national IT company that provides the highest quality products and services to serve the marketplace at competitive costs thereby achieving a high level of customer satisfaction in the rapidly changing telecom and information technology environment. To sell, implement, and maintain Internet, Intranet, and bandwidth connectivity solutions for enterprise and cable broadband users.
\end{itemize}
\textbf{Present Situtation}: Almost all countries have joined the internet, bringing significant changes to everyday life and business. The proportion of professionals with personal e-mail addresses has grown significantly, making email as essential as a fax machine once was. The demand for internet access is driven by:
\begin{itemize}
    \item \textbf{User Friendliness}: Improved interfaces allow non-technical individuals to become sophisticated users.
    \item \textbf{Universal Acess}: Commercial providers offer connections from almost any location.
    \item \textbf{Lower Cost and Cost Effectiveness}: Reduced access costs make the internet affordable, allowing businesses to realize low-cost operational improvements.
    \item \textbf{Momentum}: The increasing size of the net-wide audience attracts more information providers and businesses.
\end{itemize}

\section{Organizational Hierarchy}

The organizational structure of Subisu Cablent Pvt Ltd follows a hierarchical model with clear lines of authority and responsibility. At the top level is the Chief Executive Officer who oversees all company operations and strategic decisions. Below the CEO are several key departments including Technology and Infrastructure, Network Operations, Customer Service, Sales and Marketing, Finance and Administration, and Human Resources.

The Technology and Infrastructure department is responsible for planning and implementing network expansion, maintaining hardware infrastructure, and ensuring the reliability of technical systems. The Network Operations department, where this internship was conducted, operates the Network Operations Center and manages day-to-day network monitoring, troubleshooting, and maintenance activities. The Customer Service department handles customer inquiries, support requests, and service activation processes.

Within the Network Operations department, the structure includes the Network Operations Manager who leads the team, Senior Network Engineers who oversee technical operations, Network Engineers who handle configuration and maintenance, and NOC Technicians who monitor network performance and respond to incidents. The internship was conducted under the supervision of the Network Operations Manager, with mentorship from Senior Network Engineers and collaboration with NOC Technicians.

\section{Working Domains}

Subisu Cablent Pvt Ltd operates across several key working domains that together form a comprehensive service portfolio. The primary domain is Fiber-to-the-Home services, which involves installing and maintaining fiber optic connections to residential customers. This domain requires continuous monitoring of network performance, maintenance of last-mile connectivity, and responsive technical support to address customer issues.

Enterprise and corporate connectivity represents another important domain, where the company provides dedicated internet connections, wide area networking solutions, and secure communication channels for business customers. This domain involves designing custom network solutions, implementing service level agreements, and ensuring high availability for critical business operations.

Network infrastructure maintenance forms the backbone of all service domains, involving regular maintenance of fiber optic cables, routers, switches, and other networking equipment. This domain includes capacity planning, network optimization, and disaster recovery planning to ensure uninterrupted service delivery.

Customer technical support is a critical domain that handles all customer-facing technical activities including troubleshooting connectivity issues, activating new services, managing service upgrades, and providing technical guidance to customers. This domain requires strong communication skills alongside technical expertise to effectively resolve customer issues.

\section{Description of Intern Department/Unit}

The Network Operations Center at the Chitwan branch of Subisu Cablenet Pvt Ltd served as the primary location for this internship. The NOC functions as the central hub for network monitoring and management, operating twenty-four hours a day to ensure continuous oversight of network performance. The center is equipped with monitoring systems, diagnostic tools, and communication equipment that enable operators to detect and respond to network incidents in real-time.

The daily operations at the NOC involve continuous monitoring of network health indicators including bandwidth utilization, latency, packet loss, and connection status. Network operators analyze this data to identify potential issues before they impact customers. When incidents are detected, the team follows established procedures to diagnose problems, implement solutions, and document the resolution process.

The NOC team uses several specialized tools and systems to perform their duties effectively. Network monitoring platforms provide real-time visualization of network metrics. Ticket management systems track customer issues and resolution status. Configuration management systems store and deploy network device configurations. Remote monitoring tools enable technicians to diagnose problems at customer sites without physical visits.

The internship involved working directly under the supervision of the Branch Manager, who is also a Network Specialist. This role entailed working alongside experienced NOC staff, learning the operational procedures, understanding the monitoring tools, and participating in network troubleshooting activities. The position included observing and assisting with incident response, analyzing network performance data, and contributing to the documentation of operational procedures and best practices

\section{Literature Review / Related Study}

Effective network management is critical for ISPs to maintain service quality and operational efficiency. Network monitoring systems form the foundation of modern network management practices by providing continuous visibility into network performance and enabling proactive issue detection. According to RFC 2544, benchmarking methodologies for network interconnect devices provide standardized approaches to measure key performance metrics including throughput, latency, and packet loss, which are essential for evaluating network performance \parencite{bradner1999}.

Bandwidth management and traffic engineering are well-established practices in ISP operations. Research by \textcite{altmann2001} examines pricing models for network services and highlights the importance of understanding usage patterns for optimal resource allocation. Network traffic analysis tools such as Wireshark and tcpdump enable detailed examination of packet-level data, facilitating troubleshooting and performance optimization efforts \parencite{tanenbaum2011}.

Automated incident management systems have gained prominence in recent years as ISPs seek to improve response times and reduce manual monitoring burdens. Simple Network Management Protocol (SNMP) provides a standardized framework for monitoring network devices and collecting performance data \parencite{case1990}. Modern network management platforms integrate SNMP with other protocols and visualization tools to create comprehensive monitoring solutions \parencite{stallings2012}.

Customer support workflow optimization is another area of focus in ISP operations. Integrating network monitoring data with ticket management systems enables faster issue identification and resolution. Research on service-oriented architectures provides insights into how different systems can be integrated to create cohesive operational workflows.

Fiber optic technology and FTTH implementations have been extensively studied and documented. Understanding the characteristics of fiber optic networks, including bandwidth capabilities, signal attenuation, and deployment challenges, is essential for effective network management \parencite{keiser2016}. Best practices for FTTH network design and maintenance guide the planning and operation of modern broadband infrastructure \parencite{vandeventer2003}.
