\section{Roles and Responsibilities}

During the ten-week internship period at the Network Operations Center (NOC), several key roles and responsibilities were undertaken to ensure network stability and customer satisfaction.

\begin{itemize}
    \item \textbf{Network Monitoring:} Continuous monitoring of network health was performed using the Zabbix platform. This included observing uptime, latency, and packet loss for distribution switches in the Kathmandu Valley area.

    \item \textbf{Customer Support and CRM Management:} Responsibilities included handling the Customer Relationship Management (CRM) system to log trouble tickets. Technical issues were categorized into slow internet, loss of signal (LOS), or billing issues, and complex cases were escalated to Level-2 engineers.

    \item \textbf{Device Configuration:} Basic configuration of Layer-2 switches was carried out, including the creation of VLANs such as VLAN~10 for data traffic and VLAN~20 for voice traffic, along with port assignments using command-line interfaces.

    \item \textbf{Fiber Diagnostics:} Fiber to the Home (FTTH) issues were diagnosed by analyzing optical power levels (Rx/Tx) and identifying customers experiencing high signal attenuation below $-27$~dBm.

    \item \textbf{System Administration:} Basic Linux server administration tasks were conducted, including accessing servers via SSH and analyzing system logs to support troubleshooting processes.

    \item \textbf{Network Security Implementation:} Network security tasks included configuring Access Control Lists (ACLs) on routers and monitoring DDoS mitigation systems to safeguard core network infrastructure.
\end{itemize}

\section{Weekly Log}

\begin{longtable}{|p{1.5cm}|p{2.5cm}|p{10cm}|}
\hline
\textbf{Week} & \textbf{Date} & \textbf{Activities Performed} \\
\hline
\endfirsthead

\hline
\textbf{Week} & \textbf{Date} & \textbf{Activities Performed} \\
\hline
\endhead

First & 2082/07/23 &
\begin{minipage}[t]{10cm}
\begin{itemize}
    \item Attended internship orientation and introduction to the NOC team [cite: 8-9].
    \item Set up the workspace and installed required software tools including PuTTY, WinBox, and Microsoft Office[cite: 10].
    \item Studied company policies and security protocols related to server room access[cite: 6, 12].
\end{itemize}
\end{minipage} \\ \hline

Second & 2082/07/30 &
\begin{minipage}[t]{10cm}
\begin{itemize}
    \item Studied optical fiber hardware, differentiating between single-mode and multi-mode cables[cite: 33].
    \item Learned to identify connectors (SC, LC, and ST) and color codes (UPC and APC)[cite: 34].
    \item Observed fiber splicing operations using a fusion splicer[cite: 35].
\end{itemize}
\end{minipage} \\ \hline

Third & 2082/08/07 &
\begin{minipage}[t]{10cm}
\begin{itemize}
    \item Gained read-only access to the Zabbix monitoring dashboard[cite: 58].
    \item Monitored distribution switches and analyzed latency and packet loss graphs [cite: 58-59].
    \item Investigated critical alerts related to device downtime caused primarily by power failures[cite: 62].
\end{itemize}
\end{minipage} \\ \hline

Fourth & 2082/08/14 &
\begin{minipage}[t]{10cm}
\begin{itemize}
    \item Created trouble tickets in the CRM system based on customer reports[cite: 82].
    \item Categorized issues and escalated complex problems to Level-2 engineers[cite: 83].
    \item Learned the importance of accurate documentation for faster issue resolution[cite: 85].
\end{itemize}
\end{minipage} \\ \hline

Fifth & 2082/08/21 &
\begin{minipage}[t]{10cm}
\begin{itemize}
    \item Accessed Cisco switches using console cables and PuTTY[cite: 106].
    \item Created VLANs for data (VLAN 10) and voice (VLAN 20) traffic[cite: 107].
    \item Assigned switch ports using appropriate configuration commands like \texttt{switchport access vlan}[cite: 108].
\end{itemize}
\end{minipage} \\ \hline

Sixth & 2082/08/28 &
\begin{minipage}[t]{10cm}
\begin{itemize}
    \item Checked Rx and Tx optical power levels of customer ONUs[cite: 130].
    \item Identified customers experiencing high attenuation levels below $-27$~dBm[cite: 131].
    \item Provided guidance to customers regarding inspection of fiber patch cords for bends[cite: 131].
\end{itemize}
\end{minipage} \\ \hline

Seventh & 2082/09/06 &
\begin{minipage}[t]{10cm}
\begin{itemize}
    \item Calculated subnetting schemes for corporate clients to ensure efficient IP usage[cite: 154].
    \item Configured static routing on test routers to direct traffic[cite: 155].
    \item Resolved IP conflicts caused by incorrect port connections or loops[cite: 156, 159].
\end{itemize}
\end{minipage} \\ \hline

Eighth & 2082/09/13 &
\begin{minipage}[t]{10cm}
\begin{itemize}
    \item Accessed CentOS monitoring servers using SSH[cite: 181].
    \item Analyzed system logs using \texttt{grep}, \texttt{tail}, and \texttt{chmod}[cite: 182].
    \item Verified DNS service status using \texttt{systemctl status named}[cite: 183].
\end{itemize}
\end{minipage} \\ \hline

Ninth & 2082/09/20 &
\begin{minipage}[t]{10cm}
\begin{itemize}
    \item Studied firewall rules protecting the core network[cite: 207].
    \item Configured standard and extended ACLs to block specific IP ranges[cite: 208].
    \item Observed live DDoS mitigation procedures during attack alerts[cite: 209].
\end{itemize}
\end{minipage} \\ \hline

Tenth & 2082/09/27 &
\begin{minipage}[t]{10cm}
\begin{itemize}
    \item Compiled weekly logs into the final internship report[cite: 232].
    \item Created a network diagram summarizing internship activities[cite: 233].
    \item Presented learning outcomes to the NOC manager and supervisor[cite: 233].
\end{itemize}
\end{minipage} \\ \hline

\end{longtable}

\section{Description of the Project(s) Involved During Internship}

During the internship, practical knowledge was applied to specific technical projects aimed at improving network reliability and security.

\subsubsection{FTTH Signal Optimization and Diagnostics}
The primary technical project involved the diagnosis and optimization of Fiber to the Home (FTTH) connections to reduce customer support tickets related to "Loss of Signal" (LOS)[cite: 127, 134].

\textbf{Key Project Activities:}
\begin{itemize}
    \item \textbf{Optical Power Analysis:} Utilized Optical Line Terminal (OLT) Command Line Interfaces to query customer ONUs and retrieve real-time optical power metrics (Rx/Tx)[cite: 130].
    \item \textbf{Threshold Management:} Established that signal levels dropping below $-27$~dBm resulted in high attenuation and service instability[cite: 131].
    \item \textbf{Physical Remediation:} Guided field technicians to resolve issues identified by the data, such as cleaning dirty connectors or replacing patch cords that were bent beyond the allowable radius[cite: 131, 133].
\end{itemize}

\subsubsection{Network Security and Access Control Implementation}
The second project focused on hardening the network infrastructure against unauthorized access through the implementation of router-based security measures[cite: 203].

\textbf{Key Project Activities:}
\begin{itemize}
    \item \textbf{ACL Configuration:} Designed and deployed Standard and Extended Access Control Lists (ACLs) on routers to filter traffic and block specific malicious IP ranges[cite: 208].
    \item \textbf{Security Policies:} Implemented the security best practice of adding a default "Deny All" rule at the end of ACL sequences to ensure a fail-safe security posture[cite: 212].
    \item \textbf{Threat Monitoring:} Monitored live DDoS mitigation systems to identify and respond to incoming attack alerts on the core network[cite: 209].
\end{itemize}

\section{Tasks / Activities Performed}

\subsubsection{Network Monitoring with Zabbix}
Daily technical operations involved using the Zabbix monitoring system to ensure network uptime.
\begin{itemize}
    \item Monitored distribution switches within the Kathmandu Valley area for availability[cite: 58].
    \item Analyzed real-time graphs for latency and packet loss to detect congestion during peak traffic hours (7~PM -- 10~PM)[cite: 59, 61].
    \item Interpreted "Red" status triggers to identify device power failures at specific nodes[cite: 62].
\end{itemize}

\subsubsection{Layer-2 Switch Configuration}
Hands-on configuration was performed on Cisco switches using console connections and terminal emulators.
\begin{itemize}
    \item Established connection to switch consoles using PuTTY and serial cables[cite: 106].
    \item configured Virtual LANs (VLANs) to segregate traffic, assigning VLAN~10 for Data and VLAN~20 for Voice[cite: 107].
    \item Executed commands such as \texttt{switchport access vlan} to assign physical ports to specific logical networks and used \texttt{write memory} to save configurations[cite: 108, 111].
\end{itemize}

\subsubsection{Linux System Administration}
Server management tasks were conducted on CentOS-based infrastructure to support network services.
\begin{itemize}
    \item Accessed remote monitoring servers securely using the Secure Shell (SSH) protocol[cite: 181].
    \item Analyzed system logs located in \texttt{/var/log/messages} using text processing tools like \texttt{grep} and \texttt{tail} to identify error patterns[cite: 182, 186].
    \item Verified the status of essential network services, such as the DNS server, using the command \texttt{systemctl status named}[cite: 183].
\end{itemize}

\subsubsection{Physical Layer Troubleshooting}
Technical activities extended to the physical maintenance of the optical network.
\begin{itemize}
    \item Differentiated between Single-mode and Multi-mode fiber cables and identified connector types including SC, LC, and ST[cite: 33, 34].
    \item Diagnosed signal attenuation caused by physical stress factors such as dust contamination on connectors or excessive bending of fiber cables[cite: 37, 38].
    \item Observed and verified fiber splicing procedures using fusion splicers to ensure low-loss connections[cite: 35].
\end{itemize}
