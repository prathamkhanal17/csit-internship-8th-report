\section{Roles and Responsibilities}

During the ten-week internship at Subisu Cablent Pvt Ltd, several key roles and responsibilities were undertaken as part of the Network Operations Center team. The primary responsibility involved monitoring network performance metrics through the operational dashboard, continuously tracking bandwidth utilization, latency, packet loss, and connection status across the network. This monitoring activity required attention to detail and the ability to recognize patterns that might indicate potential issues.

Responding to network alerts and incidents formed a significant portion of the daily responsibilities. When automated systems or manual observations detected anomalies, the role included investigating the root cause, assessing the impact on services, and implementing appropriate remediation measures. This process often involved coordinating with other team members, consulting technical documentation, and applying troubleshooting procedures.

Assisting in customer technical support provided valuable experience in understanding the end-user impact of network issues. This responsibility included reviewing customer support tickets, analyzing network conditions reported by customers, and providing technical input to resolve connectivity problems. The role required effective communication with support staff to translate technical network data into actionable information for customer service teams.

Documenting network issues and resolutions was an ongoing responsibility throughout the internship. Creating detailed records of incidents, troubleshooting steps, and successful solutions contributed to the knowledge base and facilitated training of new team members. The documentation also enabled trend analysis and identification of recurring issues that required preventive measures.

Participating in bandwidth optimization projects provided opportunities to apply analytical skills to improve network efficiency. This involved collecting and analyzing bandwidth usage data, identifying peak usage patterns, and contributing to recommendations for capacity planning and resource allocation.

\section{Weekly Log}

During the first week of the internship, the focus was on orientation to the company and introduction to the Network Operations Center environment. Activities included meeting the team, understanding the organizational structure, and receiving an overview of the network infrastructure at the Chitwan branch. Training sessions covered the use of monitoring dashboards, ticket management systems, and basic operational procedures.

In the second week, the activities shifted toward shadowing experienced NOC operators during their daily monitoring routines. Learning objectives included understanding the network topology, recognizing normal performance patterns, and identifying common indicators of network issues. Hands-on practice with monitoring tools was conducted under supervision to build familiarity with the interface and data interpretation.

The third week focused on network troubleshooting fundamentals. Activities included learning router configuration basics, understanding network diagnostic commands, and practicing remote troubleshooting techniques. The intern observed and assisted with resolution of minor network incidents, following established procedures for diagnosis and resolution.

During the fourth week, bandwidth analysis tools were introduced and practical exercises were conducted. The intern learned to collect and interpret bandwidth usage data, identify peak usage periods, and analyze traffic patterns. Customer support rotation provided exposure to the service desk environment and understanding of how network issues translate to customer complaints.

Fifth week marked the beginning of the network monitoring dashboard development project. Activities included planning the dashboard features, designing the data collection mechanisms, and starting implementation of the visualization components. Weekly meetings with the mentor provided guidance on technical approach and project milestones.

The sixth week continued with dashboard development, focusing on the implementation of automated alert mechanisms. Activities included defining alert thresholds, configuring notification systems, and testing alert delivery. The intern worked on integrating data from multiple monitoring sources into a unified interface.

Seventh week activities centered on automated reporting implementation. The focus was on developing scheduled reports that summarize network performance metrics, incident statistics, and bandwidth utilization trends. Testing and refinement of reporting features was conducted to ensure accuracy and usefulness of generated reports.

During the eighth week, integration activities were undertaken to connect the monitoring dashboard with existing systems. This involved connecting to ticket management platforms, aligning data formats, and ensuring seamless information flow between systems. Documentation of the dashboard features and user guides was also completed.

Ninth week focused on performance optimization of the implemented systems. The intern conducted load testing, analyzed system performance metrics, and made adjustments to improve responsiveness and reliability. Feedback was collected from NOC staff regarding dashboard usability and feature suggestions were documented.

The final week of the internship involved completing documentation of all projects, preparing a summary presentation of achievements, and conducting knowledge transfer activities. The intern provided demonstrations of implemented systems, documented operational procedures, and ensured that all work was properly handed over to the team.

\section{Description of Projects Involved}

\subsection{Network Monitoring Dashboard Project}

The primary project undertaken during the internship was the development of a network monitoring dashboard designed to provide real-time visibility into network performance metrics. The project was motivated by the need to consolidate multiple monitoring sources into a single, intuitive interface that would enable faster detection of network issues and more efficient resource utilization.

The dashboard architecture included several key components. A data collection layer connected to existing network monitoring systems and gathered performance metrics at regular intervals. A data processing layer normalized and aggregated the collected metrics into meaningful indicators. A visualization layer presented the data through charts, graphs, and status indicators that could be easily interpreted by network operators.

Key features of the dashboard included real-time bandwidth utilization displays across different network segments, latency and packet loss visualization, geographical representation of network status, and historical trend analysis. The dashboard also incorporated user-customizable views allowing operators to focus on specific metrics or network segments relevant to their current activities.

The implementation process involved several phases starting with requirements gathering from NOC staff to understand their monitoring needs and preferences. Following this, the technical architecture was designed considering integration requirements with existing systems. Development progressed iteratively with regular reviews and feedback from users. Testing was conducted to ensure data accuracy and system reliability.

The outcomes of the project included improved visibility of network health, reduced time to detect incidents, and enhanced operational efficiency for the NOC team. Feedback from users indicated high satisfaction with the interface and improvements in their ability to monitor network status effectively.

\subsection{Bandwidth Optimization Analysis Project}

The second major project focused on analyzing bandwidth usage patterns to inform optimization strategies. The project objectives included understanding current bandwidth consumption trends, identifying peak usage periods, detecting inefficient utilization patterns, and developing recommendations for capacity planning.

Data collection involved gathering bandwidth metrics from network monitoring systems over a three-month period. The data included hourly, daily, and weekly usage patterns for different customer segments and network services. Analysis techniques included statistical analysis of usage distributions, correlation of usage with external factors such as time of day and day of week, and identification of outlier usage patterns that might indicate inefficient utilization.

Key findings from the analysis revealed clear peak usage patterns during evening hours, significant variation in usage between weekdays and weekends, and specific customer segments that demonstrated consistently high bandwidth consumption. The analysis also identified network segments where utilization was consistently low, suggesting potential for reallocation of resources.

Based on these findings, recommendations were developed for several optimization strategies. These included implementing traffic shaping policies during peak hours to ensure equitable access for all customers, promoting off-peak usage through pricing incentives, and planning capacity expansion for segments approaching utilization thresholds. The analysis also informed discussions about service tier structures that could better match customer needs.

\section{Detailed Technical Tasks and Activities}

\subsection{Router Configuration and Network Management}

Throughout the internship, router configuration activities provided valuable experience in network management practices. Working with Mikrotik and Cisco router platforms, tasks included configuring basic network interfaces, setting up routing protocols, implementing access control lists, and establishing firewall rules. These activities required understanding of networking fundamentals including IP addressing, subnet masks, routing tables, and network security principles.

Configuration management procedures were learned and applied, including documenting changes before implementation, testing configurations in lab environments before deployment, and maintaining backup configurations for quick recovery in case of errors. The intern gained experience with both command-line interfaces and graphical configuration tools, understanding the advantages and appropriate use cases for each approach.

Advanced configuration tasks included configuring virtual private networks, implementing quality of service policies to prioritize critical traffic, and setting up network address translation for connecting internal networks to external services. These activities required understanding of complex networking concepts and careful attention to detail to avoid disrupting network services.

\subsection{Network Traffic Analysis}

Network traffic analysis formed a significant component of technical activities during the internship. Using tools such as Wireshark, the intern learned to capture and analyze network packets at various points in the network. Analysis tasks included identifying traffic types, detecting unusual patterns that might indicate security incidents or performance problems, and troubleshooting connectivity issues by examining packet-level details.

Traffic analysis provided insights into protocol usage, bandwidth consumption by different applications, and communication patterns between network devices. The intern developed skills in applying filters to focus on specific types of traffic, interpreting packet headers and payloads, and correlating traffic patterns with observed network behavior.

Advanced analysis techniques included flow analysis using tools such as NetFlow to understand traffic patterns at a higher level of aggregation, identifying the source and destination of large traffic volumes, and detecting denial of service indicators through unusual traffic patterns. These skills are essential for effective network troubleshooting and security monitoring.

\subsection{Fiber Optic Link Testing and Troubleshooting}

Working with fiber optic infrastructure provided practical experience in the physical layer of network operations. Tasks included testing fiber optic links using optical time-domain reflectometer equipment, measuring signal strength and loss, and identifying faults in fiber cables. The intern learned to interpret test results, locate the position of faults, and coordinate repair activities with field technicians.

Fiber optic troubleshooting procedures included identifying common failure modes such as cable cuts, connector contamination, and signal degradation over distance. The intern gained understanding of fiber optic network architecture including backbone connections, distribution networks, and last-mile connections to customer premises.

Documentation of fiber optic test results and repair activities contributed to the maintenance of network asset records and facilitated planning of network maintenance activities. Understanding the characteristics of fiber optic technology including bandwidth capabilities, susceptibility to physical damage, and environmental considerations provided context for planning network reliability improvements.

\subsection{Customer Network Setup and Configuration}

Customer-facing technical activities included assisting with network setup and configuration for new service activations. Tasks included configuring customer premise equipment such as routers and modems, establishing wireless network configurations, and verifying connectivity to the ISP network. These activities required understanding of various customer network environments and the ability to adapt configurations to meet specific customer needs.

The intern participated in remote troubleshooting of customer connectivity issues, using diagnostic tools to identify problems and guiding customers through resolution steps. This required strong communication skills to translate technical procedures into clear instructions for customers with varying levels of technical expertise.

Network security considerations were integrated into customer setup activities, including implementing appropriate password policies, configuring firewall rules to protect customer networks, and educating customers about security best practices. Documenting customer configurations and support interactions contributed to the knowledge base and facilitated future support activities.

\subsection{Monitoring System Setup and Maintenance}

Setting up and maintaining monitoring systems was a core technical activity throughout the internship. Tasks included configuring monitoring agents on network devices, defining performance metrics to be collected, establishing thresholds for alert generation, and customizing dashboard displays to meet user preferences.

The intern learned to calibrate monitoring systems to avoid false alerts while ensuring genuine issues were detected promptly. This involved analyzing historical performance data to establish appropriate thresholds, testing alert conditions with controlled scenarios, and refining configurations based on operational experience.

System maintenance activities included regular reviews of monitoring configuration to ensure alignment with network changes, updating device inventories as equipment was added or removed, and verifying data accuracy through periodic audits. These activities ensure that monitoring systems continue to provide reliable and actionable information.

\subsection{Incident Documentation and Ticket Management}

Documentation of network incidents and management of support tickets formed an important part of operational activities. The intern learned to create detailed incident records including timestamps, symptoms observed, diagnostic steps taken, resolution implemented, and lessons learned. This documentation enables trend analysis, facilitates training, and provides context for future incidents.

Ticket management activities involved creating and updating support tickets for customer issues, tracking resolution status, escalating complex issues appropriately, and closing tickets when resolution was confirmed. Understanding the lifecycle of support tickets and the importance of timely communication with customers contributed to effective customer service delivery.

The intern gained experience with ticket management systems including entering technical details, assigning priority levels, attaching relevant documentation such as network logs and test results, and generating reports on ticket statistics. These skills are essential for maintaining organized support operations and measuring service quality.

\subsection{Network Security Basics}

Introduction to network security concepts provided foundation for understanding security considerations in ISP operations. Topics studied included firewall principles, access control mechanisms, virtual private network technologies, and network monitoring for security purposes. The intern gained understanding of security policies implemented at Subisu Cablent to protect network infrastructure and customer data.

Practical security activities included configuring firewall rules to control traffic flow, implementing intrusion detection systems to identify potential security threats, and conducting security audits to verify that configurations complied with security policies. These activities provided awareness of the importance of security in network operations and introduced tools and techniques for maintaining network security.

The intern learned about common security threats facing ISPs including distributed denial of service attacks, phishing attempts targeting customers, and unauthorized access attempts. Understanding these threats informed security practices and contributed to a security-conscious approach to all operational activities.
