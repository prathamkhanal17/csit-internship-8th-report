\chapter{INTERNSHIP ACTIVITIES}
\section{Roles and Responsibilities}

During the ten-week internship period at the Network Operations Center (NOC), several key roles and responsibilities were undertaken to ensure network stability and customer satisfaction.

\begin{itemize}
    \item \textbf{Network Monitoring:} Continuous monitoring of network health was performed using the Zabbix platform \parencite{zabbix2023}. This included observing uptime, latency, and packet loss for distribution switches in the Chitwan area.

    \item \textbf{Customer Support and CRM Management:} Responsibilities included handling the Customer Relationship Management (CRM) system to log trouble tickets. Technical issues were categorized into slow internet, loss of signal (LOS), or billing issues, effectively managing the incident lifecycle \parencite{adams2021}.

    \item \textbf{Device Configuration:} Basic configuration of Layer-2 switches was carried out, including the creation of VLANs such as VLAN~10 for data traffic and VLAN~20 for voice traffic \parencite{cisco2022}, along with port assignments using command-line interfaces.

    \item \textbf{Fiber Diagnostics:} Fiber to the Home (FTTH) issues were diagnosed by analyzing optical power levels (Rx/Tx) and identifying customers experiencing high signal attenuation below the standard threshold of $-27$~dBm \parencite{itu2021}.

    \item \textbf{System Administration:} Basic Linux server administration tasks were conducted, including accessing servers via SSH and analyzing system logs to support troubleshooting processes \parencite{redhat2022}.

    \item \textbf{Network Security Implementation:} Network security tasks included configuring Access Control Lists (ACLs) on routers and monitoring DDoS mitigation systems to safeguard core network infrastructure \parencite{stallings2020}.
\end{itemize}

\section{Weekly Log}

\begin{longtable}{|p{1.5cm}|p{2.5cm}|p{10cm}|}
\caption{Weekly Internship Activities Log} \\
\hline
\textbf{Week} & \textbf{Date} & \textbf{Activities Performed} \\
\hline
\endfirsthead

\hline
\textbf{Week} & \textbf{Date} & \textbf{Activities Performed} \\
\hline
\endhead

First & 2082/07/23 &
\begin{minipage}[t]{10cm}
\begin{itemize}
    \item Attended internship orientation and introduction to the NOC team.
    \item Set up the workspace and installed required software tools including PuTTY, WinBox, and Microsoft Office.
    \item Studied company policies and security protocols related to server room access and physical security \parencite{stallings2020}.
\end{itemize}
\end{minipage} \\ \hline

Second & 2082/07/30 &
\begin{minipage}[t]{10cm}
\begin{itemize}
    \item Studied optical fiber hardware, differentiating between single-mode and multi-mode cables \parencite{keiser2016}.
    \item Learned to identify connectors (SC, LC, and ST) and color codes (UPC and APC) \parencite{govind2020}.
    \item Observed fiber splicing operations using a fusion splicer to ensure low insertion loss \parencite{itu2021}.
\end{itemize}
\end{minipage} \\ \hline

Third & 2082/08/07 &
\begin{minipage}[t]{10cm}
\begin{itemize}
    \item Gained read-only access to the Zabbix monitoring dashboard \parencite{zabbix2023}.
    \item Monitored distribution switches and analyzed latency and packet loss graphs.
    \item Investigated critical alerts related to device downtime caused primarily by power failures.
\end{itemize}
\end{minipage} \\ \hline

Fourth & 2082/08/14 &
\begin{minipage}[t]{10cm}
\begin{itemize}
    \item Created trouble tickets in the CRM system based on customer reports.
    \item Categorized issues and escalated complex problems to Level-2 engineers \parencite{adams2021}.
    \item Learned the importance of accurate documentation for faster issue resolution \parencite{adams2021}.
\end{itemize}
\end{minipage} \\ \hline

Fifth & 2082/08/21 &
\begin{minipage}[t]{10cm}
\begin{itemize}
    \item Accessed Cisco switches using console cables and PuTTY.
    \item Created VLANs for data (VLAN 10) and voice (VLAN 20) traffic \parencite{cisco2022}.
    \item Assigned switch ports using appropriate configuration commands like \texttt{switchport access vlan} \parencite{cisco2022}.
\end{itemize}
\end{minipage} \\ \hline

Sixth & 2082/08/28 &
\begin{minipage}[t]{10cm}
\begin{itemize}
    \item Checked Rx and Tx optical power levels of customer ONUs.
    \item Identified customers experiencing high attenuation levels below the $-27$~dBm threshold \parencite{itu2021}.
    \item Provided guidance to customers regarding inspection of fiber patch cords for bends.
\end{itemize}
\end{minipage} \\ \hline

Seventh & 2082/09/06 &
\begin{minipage}[t]{10cm}
\begin{itemize}
    \item Calculated subnetting schemes for corporate clients to ensure efficient IP usage \parencite{forouzan2021}.
    \item Configured static routing on test routers to direct traffic.
    \item Resolved IP conflicts caused by incorrect port connections or loops \parencite{forouzan2021}.
\end{itemize}
\end{minipage} \\ \hline

Eighth & 2082/09/13 &
\begin{minipage}[t]{10cm}
\begin{itemize}
    \item Accessed CentOS monitoring servers using SSH.
    \item Analyzed system logs using \texttt{grep}, \texttt{tail}, and \texttt{chmod} \parencite{redhat2022}.
    \item Verified DNS service status using \texttt{systemctl status named} \parencite{redhat2022}.
\end{itemize}
\end{minipage} \\ \hline

Ninth & 2082/09/20 &
\begin{minipage}[t]{10cm}
\begin{itemize}
    \item Studied firewall rules protecting the core network \parencite{stallings2020}.
    \item Configured standard and extended ACLs to block specific IP ranges \parencite{stallings2020}.
    \item Observed live DDoS mitigation procedures during attack alerts.
\end{itemize}
\end{minipage} \\ \hline

Tenth & 2082/09/27 &
\begin{minipage}[t]{10cm}
\begin{itemize}
    \item Compiled weekly logs into the final internship report.
\end{itemize}
\end{minipage} \\ \hline

\end{longtable}

\section{Description of Projects Involved}

During the internship, practical knowledge was applied to specific technical projects aimed at improving network reliability and security.

\subsubsection{FTTH Signal Optimization and Diagnostics}
The primary technical project involved the diagnosis and optimization of Fiber to the Home (FTTH) connections to reduce customer support tickets related to "Loss of Signal" (LOS).

\textbf{Key Project Activities:}
\begin{itemize}
    \item \textbf{Optical Power Analysis:} Utilized Optical Line Terminal (OLT) Command Line Interfaces to query customer ONUs and retrieve real-time optical power metrics (Rx/Tx).
    \item \textbf{Threshold Management:} Established that signal levels dropping below $-27$~dBm resulted in high attenuation and service instability, consistent with GPON standards \parencite{itu2021}.
    \item \textbf{Physical Remediation:} Guided field technicians to resolve issues identified by the data, such as cleaning dirty connectors or replacing patch cords that were bent beyond the allowable radius \parencite{govind2020}.
\end{itemize}

\subsubsection{Network Security and Access Control Implementation}
The second project focused on hardening the network infrastructure against unauthorized access through the implementation of router-based security measures.

\textbf{Key Project Activities:}
\begin{itemize}
    \item \textbf{ACL Configuration:} Designed and deployed Standard and Extended Access Control Lists (ACLs) on routers to filter traffic and block specific malicious IP ranges \parencite{stallings2020}.
    \item \textbf{Security Policies:} Implemented the security best practice of adding a default "Deny All" rule at the end of ACL sequences to ensure a fail-safe security posture \parencite{stallings2020}.
    \item \textbf{Threat Monitoring:} Monitored live DDoS mitigation systems to identify and respond to incoming attack alerts on the core network.
\end{itemize}

\section{Detailed Technical Tasks/Activities}

\subsubsection{Network Monitoring with Zabbix}
Daily technical operations involved using the Zabbix monitoring system to ensure continuous network uptime and service reliability.
\begin{itemize}
    \item Monitored distribution switches deployed across the Chitwan area to verify device availability and link status \parencite{zabbix2023}.
    \item Observed key performance indicators such as CPU utilization, memory usage, interface bandwidth, latency, and packet loss through real-time and historical graphs.
    \item Analyzed traffic patterns during peak hours (7~PM -- 10~PM) to identify congestion issues and abnormal spikes in bandwidth usage.
    \item Interpreted critical ``Red'' status triggers generated by Zabbix to promptly identify device outages, power failures, or link-down events at specific network nodes \parencite{zabbix2023}.
    \item Assisted in acknowledging alerts and escalating critical incidents to senior network engineers for timely resolution.
\end{itemize}

\subsubsection{Layer-2 Switch Configuration}
Hands-on configuration tasks were performed on Cisco Layer-2 switches using console connections and terminal emulation tools.
\begin{itemize}
    \item Established console access to switches using serial cables and PuTTY to perform initial configuration and troubleshooting.
    \item Configured Virtual LANs (VLANs) to logically segregate network traffic, assigning VLAN~10 for Data services and VLAN~20 for Voice services \parencite{cisco2022}.
    \item Assigned physical switch ports to appropriate VLANs using commands such as \texttt{switchport mode access} and \texttt{switchport access vlan}.
    \item Verified VLAN configurations using commands like \texttt{show vlan brief} and tested connectivity between end devices.
    \item Saved and backed up switch configurations using the \texttt{write memory} command to ensure persistence after device reboot \parencite{cisco2022}.
\end{itemize}

\subsubsection{Linux System Administration}
Server management and monitoring tasks were conducted on CentOS-based systems that supported core network services.
\begin{itemize}
    \item Accessed remote monitoring and DNS servers securely using the Secure Shell (SSH) protocol.
    \item Examined system and service logs located in \texttt{/var/log/messages} to identify warnings, errors, and unusual behavior using tools such as \texttt{grep}, \texttt{tail}, and \texttt{less} \parencite{redhat2022}.
    \item Verified the operational status of essential network services, including DNS, using the command \texttt{systemctl status named} \parencite{redhat2022}.
    \item Restarted services when required and confirmed service recovery while ensuring minimal disruption to network operations.
    \item Monitored disk usage, memory consumption, and system uptime to maintain overall server health.
\end{itemize}

\subsubsection{Physical Layer Troubleshooting}
Technical responsibilities also included physical layer inspection and maintenance of the optical fiber network.
\begin{itemize}
    \item Identified and differentiated between Single-mode and Multi-mode fiber optic cables based on core size, color coding, and application scenarios \parencite{keiser2016}.
    \item Recognized common fiber connector types such as SC, LC, and ST, and understood their usage in access and distribution networks \parencite{govind2020}.
    \item Diagnosed signal attenuation issues caused by dust contamination, improper connector mating, excessive bending, or mechanical stress on fiber cables.
    \item Observed fiber splicing operations using fusion splicers and verified splice quality to ensure low insertion loss and minimal signal reflection \parencite{itu2021}.
    \item Assisted in basic optical link inspection and maintenance procedures in accordance with industry best practices.
\end{itemize}
