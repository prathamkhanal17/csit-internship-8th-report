\chapter{INTRODUCTION}
\section{Introduction}

This report documents the internship activities undertaken in the role of Network Operations and ISP Support. The internship spanned a period of ten weeks, focusing on the practical application of networking concepts within a live Internet Service Provider (ISP) environment. The work was primarily conducted in a Network Operations Center (NOC) that operates on a 24/7 shift rotation. The core of the internship involved bridging the gap between theoretical knowledge and professional practice, covering areas such as physical layer connectivity, network monitoring, and system administration.

The project involved working with real-time network data, analyzing bandwidth usage patterns, and developing automated mechanisms for incident detection and response. The internship spanned from June 2024 to August 2024 at the Chitwan branch of Subisu Cablent Ltd. The primary objective was to enhance the visibility of network health metrics, optimize bandwidth allocation strategies, and improve the efficiency of customer support workflows through systematic monitoring and documentation processes.

\section{Problem Statement}

Modern ISPs face constant challenges regarding network uptime, signal integrity, and security. During this internship, several key technical problems were addressed:

\begin{itemize}
    \item \textbf{Signal Attenuation} - Addressing signal loss caused by physical factors such as dust on connectors or fiber cables bent beyond their bend radius.
    \item \textbf{Network Congestion and Downtime} - Diagnosing high latency during peak traffic hours (7 PM – 10 PM) and identifying device downtime caused by power failures.
    \item \textbf{Connectivity Disputes} - Resolving IP conflict issues where multiple devices claimed the same address and troubleshooting "Loss of Signal" (LOS) errors for end-users
    \item \textbf{Security Threats} - Mitigating risks related to unauthorized access and potential DDoS attacks on the core network.
\end{itemize}

\section{Objectives}

The primary objective of this internship was to gain hands-on experience in maintaining and troubleshooting a large-scale ISP network. The specific objectives were:

\begin{itemize}
    \item To gain general exposure to network infrastructure, connectivity standards, and technical troubleshooting procedures.
    \item To support daily operational workflows, including system monitoring and administrative process management.
\end{itemize}

\section{Scope and Limitation}

\begin{itemize}
    \item \textbf{Scope}: The scope of work encompassed the entire network stack, from the physical layer (fusion splicing and cabling) to the application layer (DNS and web-based monitoring). It included both technical backend configurations and frontend customer support interaction.
    \item \textbf{Limitations}: Due to the critical nature of the live ISP network, certain high-risk configurations were restricted. For instance, access to the Zabbix monitoring dashboard was limited to "read-only" permissions. Additionally, initial configuration practice for switches and routers was conducted on lab devices rather than live customer equipment to prevent service disruption. Complex technical issues were escalated to engineers rather than being resolved independently.
\end{itemize}

\section{Report Organization}
The report follows a well-defined structure that systematically presents the internship work, beginning with an introduction and progressing through organizational background, internship activities, and conclusions, followed by references, bibliography, and appendices.
\begin{itemize}

  \item \textbf{Chapter 1:} Provides an introduction to the internship project, including the problem statement, objectives, scope, and limitations.

  \item \textbf{Chapter 2:} Presents detailed information about the organization, including the company profile, organizational hierarchy, working domains, and the department where the internship was conducted. This chapter also includes a literature review of relevant technologies and best practices in ISP network management.

  \item \textbf{Chapter 3:} Details the internship activities, including roles and responsibilities, a weekly log of technical activities, descriptions of projects involved, and the technical tasks performed during the ten-week period.

  \item \textbf{Chapter 4:} Presents the conclusion of the internship, summarizing achievements and discussing learning outcomes.

  \item \textbf{References:} Contains APA-style citations of all works referenced in the report.

  \item \textbf{Appendices:} Includes network diagrams, screenshots of monitoring tools, detailed work logs, and relevant source code snippets.
\end{itemize}
