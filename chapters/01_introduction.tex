\section{Introduction to the Project}

Internet Service Providers play a critical role in delivering reliable connectivity services to customers across various sectors. The efficient management of network infrastructure, monitoring of performance metrics, and timely resolution of technical issues are essential for maintaining service quality. This internship project focused on developing and implementing a network monitoring system within the Network Operations Center of Subisu Cablent Pvt Ltd. The ten-week program provided hands-on experience in ISP operations, network troubleshooting, and the application of network management tools.

The project involved working with real-time network data, analyzing bandwidth usage patterns, and developing automated mechanisms for incident detection and response. The internship spanned from June 2024 to August 2024 at the Chitwan branch of Subisu Cablent Pvt Ltd. The primary objective was to enhance the visibility of network health metrics, optimize bandwidth allocation strategies, and improve the efficiency of customer support workflows through systematic monitoring and documentation processes.

\section{Problem Statement}

Modern ISPs face significant challenges in maintaining optimal network performance while ensuring high service quality for customers. At Subisu Cablent Pvt Ltd, several operational issues were identified that needed to be addressed during the internship period. The existing network monitoring capabilities had limitations in detecting network incidents in real-time, resulting in delayed response times to network outages and performance degradation.

The lack of automated alert systems meant that network operators had to manually monitor multiple dashboards and logs, increasing the risk of overlooking critical issues. Bandwidth utilization patterns were not systematically analyzed, leading to inefficient resource allocation during peak usage periods. Customer support workflows relied heavily on manual processes, which caused delays in issue resolution and affected customer satisfaction. The absence of centralized documentation for network incidents made it difficult to track recurring problems and implement preventive measures effectively.

\section{Objectives}

The internship project had several clearly defined objectives aimed at addressing the identified problems and improving overall network management capabilities. The primary objective was to develop a network monitoring dashboard that provides real-time visibility into network performance metrics including bandwidth usage, latency, packet loss, and connection status.

The second objective focused on implementing an automated alert system that notifies network operators of critical incidents as soon as they are detected, reducing response times and minimizing service disruption. A third objective involved conducting comprehensive analysis of bandwidth usage patterns to identify peak usage times, understand consumption trends, and recommend optimization strategies for efficient resource allocation.

The fourth objective centered on improving customer support workflows by integrating monitoring data with ticket management systems, enabling faster issue identification and resolution. The final objective aimed at documenting all network incidents, troubleshooting procedures, and best practices to create a knowledge base for training and future reference.

\section{Scope and Limitations}

The internship project had a well-defined scope that focused on specific aspects of network operations within the Network Operations Center. The project covered network monitoring implementation, bandwidth analysis and optimization, technical support workflow improvement, and documentation of operational procedures. The technical activities included configuring monitoring tools, analyzing network traffic data, setting up automated alerts, and assisting with router configuration and troubleshooting.

However, several limitations were encountered during the internship period. Access to certain network segments was restricted due to security policies, limiting the scope of monitoring and testing activities. The ten-week duration imposed time constraints on comprehensive implementation and testing of all proposed solutions. Additional time was required for training and orientation to the existing network infrastructure, tools, and operational procedures.

The internship did not involve direct customer-facing activities or sales operations, as the focus was specifically on technical operations and network management. Complex network architectural changes and major infrastructure upgrades were outside the scope of the internship project and would require long-term planning and approval processes.

\section{Report Organization}

This report is organized into four main chapters followed by references, bibliography, and appendices. Chapter 1 provides an introduction to the internship project, including the problem statement, objectives, scope, and limitations. Chapter 2 presents detailed information about the organization including company profile, organizational hierarchy, working domains, and the specific department where the internship was conducted. This chapter also includes a literature review of relevant technologies and best practices in ISP network management.

Chapter 3 details the internship activities, including roles and responsibilities, a weekly log of technical activities, description of projects involved, and detailed technical tasks performed during the ten-week period. Chapter 4 presents the conclusion of the internship, summarizing achievements and discussing learning outcomes. The references section contains APA-style citations of all works referenced in the report. The bibliography lists additional resources studied during the internship. The appendices include network diagrams, screenshots of monitoring tools, detailed work logs, and relevant source code snippets.
