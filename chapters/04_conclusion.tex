\section{Conclusion}

The ten-week internship at the Network Operations Center (NOC) of Subisu Cablenet Pvt. Ltd.\ served as a crucial bridge between academic theoretical knowledge and professional industry practice. Over the course of the internship, the transition from understanding networking concepts in a classroom to applying them in a live and critical ISP environment was successfully achieved.

The experience provided deep insights into the operational complexities of a major Internet Service Provider. Working in a 24/7 shift environment highlighted the importance of continuous monitoring and rapid incident response. The progression from basic orientation to handling complex tasks—such as configuring VLANs on Cisco switches, troubleshooting fiber attenuation issues, and implementing security ACLs—demonstrated significant growth in technical competency \parencite{cisco2022,itu2021,stallings2020}.

Furthermore, the internship emphasized that technical skills alone are insufficient; effective communication and detailed documentation are equally vital. The responsibility of diagnosing loss of signal (LOS) issues and guiding customers through troubleshooting steps enhanced the ability to translate complex technical concepts into clear and actionable instructions \parencite{adams2021,kumar2022}.

In conclusion, this internship established a strong foundation in network operations and system administration. It provided the confidence to work with enterprise-grade network equipment and monitoring tools, thereby fulfilling the objectives of the internship program and preparing for a future career in network engineering.

\section{Learning Outcomes}

The internship resulted in the acquisition of both technical and professional competencies. The key learning outcomes are categorized as follows.

\subsection{Technical Competencies}

\begin{itemize}
    \item \textbf{Optical Network Expertise:} Gained practical proficiency in handling optical fiber hardware, including differentiating between single-mode and multi-mode cables, identifying standard connectors (SC, LC, and ST), and understanding the impact of physical stress such as bending and dust on signal attenuation \parencite{itu2021,govind2020}.

    \item \textbf{Network Infrastructure Configuration:} Developed the ability to configure Layer-2 and Layer-3 network devices. Key skills included creating and managing VLANs to segregate voice and data traffic, calculating IP subnets for corporate clients, and configuring static routes for efficient traffic forwarding \parencite{cisco2022,forouzan2021}.

    \item \textbf{System Administration and Linux Proficiency:} Acquired working knowledge of Linux-based server environments (CentOS). Learned to utilize command-line tools such as \texttt{grep} and \texttt{tail} for system log analysis and manage services using \texttt{systemctl} \parencite{redhat2022}.

    \item \textbf{Network Monitoring and Diagnostics:} Developed proficiency in using network monitoring tools such as Zabbix. Learned to interpret real-time performance metrics related to latency, packet loss, and device availability to proactively identify faults and power-related outages \parencite{zabbix2023}.

    \item \textbf{Security Implementation:} Gained foundational knowledge of network security by designing and implementing Access Control Lists (ACLs) and understanding the roles of firewalls and DDoS mitigation systems in protecting core network infrastructure \parencite{stallings2020}.
\end{itemize}

\subsection{Professional and Soft Skills}

\begin{itemize}
    \item \textbf{Incident Management:} Learned the complete lifecycle of trouble tickets within a CRM system, including ticket creation, categorization, prioritization, escalation, and resolution based on issue severity \parencite{adams2021}.

    \item \textbf{Problem Solving Under Pressure:} Developed the ability to analyze and resolve network issues efficiently in a live NOC environment, where minimizing customer downtime was critical.

    \item \textbf{Technical Documentation:} Recognized the importance of maintaining accurate technical documentation and configuration backups. Proper documentation and saving device configurations were found to be essential for knowledge transfer and preventing configuration loss \parencite{cisco2022}.
\end{itemize}
