\chapter{Network Diagrams}

\section{Network Topology Overview}

Figure A.1 illustrates the high-level network topology of Subisu Cablent operations in the Chitwan region. The diagram shows the interconnection between backbone network links, metropolitan area network infrastructure, and last-mile connections to customer premises. The topology includes redundancy configurations designed to ensure service availability in case of equipment failures.

\section{Core Network Architecture}

The core network consists of multiple high-capacity routers connected in a mesh configuration providing redundancy and load balancing. Backbone links utilize fiber optic connections with capacity in the range of multiple gigabits per second. The core network provides connectivity to regional distribution points and serves as the aggregation point for all customer traffic.

\section{Distribution Network Architecture}

Distribution nodes connect the core network to the last-mile infrastructure serving customers. These nodes perform traffic aggregation and apply quality of service policies to prioritize critical applications. Distribution network equipment includes layer 3 switches and routers configured with virtual private network capabilities to support enterprise customers.

\chapter{Monitoring Tool Screenshots}

\section{Network Monitoring Dashboard}

Figure B.1 displays the main network monitoring dashboard developed during the internship. The dashboard provides real-time visualization of bandwidth utilization, latency metrics, and connection status across monitored network segments. Color-coded indicators enable quick assessment of network health.

\section{Bandwidth Analysis Interface}

Figure B.2 shows the bandwidth analysis interface displaying historical usage patterns. The interface enables filtering by time period, network segment, and customer segment. Interactive charts allow identification of peak usage periods and unusual consumption patterns.

\section{Alert Configuration Screen}

Figure B.3 presents the alert configuration interface where thresholds for automated notifications are defined. Users can configure different alert levels for various metrics and specify notification delivery methods including email, SMS, and dashboard notifications.

\chapter{Work Logs}

\section{Daily Activity Log}

The daily activity log records tasks performed during each day of the internship. Activities include network monitoring sessions, troubleshooting activities attended, configuration changes made, and documentation completed. The log includes timestamps and duration estimates for each activity.

\section{Incident Summary}

The incident summary provides a consolidated view of all network incidents observed and addressed during the internship period. Each incident entry includes the time of detection, symptoms observed, diagnostic steps taken, resolution implemented, and affected customers.

\section{Weekly Summary Reports}

Weekly summary reports provide high-level overview of activities completed each week including project progress, skills developed, challenges encountered, and achievements. These reports formed the basis for weekly feedback meetings with the internship mentor.

\chapter{Source Code}

\section{Dashboard Data Collection Script}

The dashboard data collection script gathers performance metrics from network monitoring systems and stores them in a database for analysis and visualization. The script implements error handling and retry logic to ensure reliable data collection.

\section{Alert Notification Module}

The alert notification module processes threshold violations and generates notifications according to configured preferences. The module implements priority queuing to ensure critical alerts are delivered immediately while non-critical alerts are batched for efficiency.

\section{Bandwidth Analysis Functions}

Bandwidth analysis functions process raw usage data and generate statistical summaries including mean, median, maximum, and minimum utilization for defined time periods. These functions also detect trends and anomalies that may require investigation.

\section{Configuration Backup Script}

The configuration backup script retrieves current configurations from network devices and stores them with timestamps for disaster recovery purposes. The script can restore configurations to previous states in case of configuration errors or device failures.
